

\documentclass[a4paper,10pt]{article}

\usepackage[utf8]{inputenc}
\usepackage[italian]{babel}
\usepackage{amsmath}
\usepackage{amsthm}
\usepackage{fancyhdr}
\usepackage{amsfonts}
\usepackage{amssymb}
\usepackage[parfill]{parskip}
\usepackage{multicol}
\usepackage{fullpage}

\topmargin -1cm
\textheight 26cm
\oddsidemargin -1cm
\textwidth 18cm

\title{Lupus in tempo reale}
\author{Settima edizione}
\date{28 febbraio 2018}

\begin{document}
\pagenumbering{gobble}

\maketitle

\emph{È vietato mostrare il contenuto di questa lettera ad altre persone, fino al termine della partita.}


\subsection*{Ruolo e personaggio}
Carissim{{ player.oa }} {{ player.full_name }}, nel corso della partita ricoprirai il ruolo di {{ player.role.name }}.




    
        {{ message }}
    



    Ti comunichiamo inoltre che sei stato nominato Sindaco del villaggio.



\subsection*{Composizione del villaggio}
La composizione iniziale del villaggio non è nota.
 %
Tuttavia vi vengono fornite le seguenti utilissime proposizioni, che i Game Master garantiscono essere vere.
\nopagebreak
\begin{itemize}

  \item {{ proposition.text }}

\end{itemize}


Il Sindaco del villaggio è {{ mayor.full_name }}.


\subsection*{Credenziali di accesso all'interfaccia web}

Le comunicazioni relative alle dinamiche di gioco (non le comunicazioni tra giocatori, bensì le azioni eseguite, le votazioni, ecc.) avvengono attraverso l'interfaccia web di \emph{Lupus in tempo reale}.
L'indirizzo a cui collegarsi è il seguente:
\nopagebreak
\begin{center}
    \verb|http://lupus.uz.sns.it/|
\end{center}

Le credenziali con cui puoi accedere all'area riservata (per votare, utilizzare il tuo potere notturno e visualizzare le informazioni personali) sono le seguenti.
\nopagebreak
\begin{center}
\begin{tabular}{ll}
    Username: & \verb| {{ player.user.username }} | \\
    Password: & \verb| {{ password }} | \\
\end{tabular}
\end{center}


\subsection*{Elenco dei giocatori}
Partecipano alla partita {{ numplayers }} giocatori, di cui si fornisce l'elenco completo:
\nopagebreak

\begin{minipage}{0.5\textwidth}
\small
\begin{tabular}{ll}

        {{ player.full_name }} & \verb| {{ player.user.email }} | \\

\end{tabular}
\end{minipage}
\begin{minipage}{0.5\textwidth}
\small
\begin{tabular}{ll}


\end{tabular}
\end{minipage}

È stata creata una mailing-list che potete utilizzare per scrivere a tutti i giocatori:
\begin{center}
\verb||
\end{center}


\subsection*{Segretezza delle informazioni}

Come già specificato in precedenza, è vietato mostrare il contenuto di questa lettera a qualsiasi persona prima che la partita sia conclusa, anche se il tuo personaggio muore o viene esiliato.
Ciò non significa che tu non possa \emph{dichiarare} di avere un determinato ruolo: semplicemente non puoi dimostrarlo facendo vedere questa lettera a qualcun altro.

Discorso analogo si applica all'interfaccia web: alcune pagine (quelle che, nel menu, appaiono sotto la scritta ``Area riservata'') contengono informazioni che possono essere viste solo da te. Pertanto assicurati di essere lontano da occhi indiscreti ogni volta che accedi a pagine dell'area riservata.


\subsection*{Comunicazioni con i Game Master}

I Game Master che gestiscono la partita sono Alessandro Iraci, Matteo Migliorini e Manuele Cusumano.
In caso di dubbi o necessità, potete contattarli all'indirizzo e-mail:

\end{document}




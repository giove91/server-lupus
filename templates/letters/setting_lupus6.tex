

\documentclass{letter}

\usepackage[utf8]{inputenc}
\usepackage[italian]{babel}


\signature{Alessandro Iraci}
\date{17 novembre 2015}
\begin{document}

\begin{letter}{}
\opening{Gentili concittadini,}

ma soprattutto gentilissim{{ player.oa }} {{ player.full_name }},

affranto dal dolore vi comunico che la sicurezza del nostro villaggio è stata compromessa nuovamente. Gli ululati che riecheggiano nella notte non appartengono solo ai cani che normalmente riposano sotto il Carducci, ma a bestie ben più pericolose. Vecchi lupi sono fuggiti in Scozia, ma nuove fiere hanno preso il loro posto.

Siamo stati capaci di sommergere il Sant'Anna con i nostri gavettoni, ottenere il miglior medagliere alle XCool ed eliminare i Generals, ma ci siamo trovati in difficoltà contro la nuova invasione di lupi mannari. Allo scopo di risolvere il problema, sono andato a richiamare dalla California Giovanni Mascellani, Sei Volte Vincitore della 24 Ore, e Giovanni Paolini, Premio Oscar al Miglior Trailer 2012, 2013, 2014 e 2015. Dopo giorni e giorni di intense riflessioni, abbiamo concluso che la migliore soluzione possibile è trasferirsi da una mia vecchia amica su un isolotto del Pacifico, e ragionare su come salvare il villaggio sorseggiando una pi\~na colada.

Sono sicuro che nel frattempo voi, valorosi concittadini, sarete capaci di gestire la situazione con la solita sicurezza. Affido le chiavi della città a {{ mayor.full_name }}, scelto per il valore dimostrato nell'ultimo torneo di bocce.

Un abbraccio dal vostro affettuoso ex Sindaco,

\bigskip

\noindent Alessandro Iraci

\end{letter}

\end{document}





\documentclass{letter}

\usepackage[utf8]{inputenc}
\usepackage[italian]{babel}


\signature{Alessandro Iraci}
\date{26 novembre 2018}
\begin{document}

\begin{letter}{}
\opening{Gentili concittadini,}

ma soprattutto gentilissim{{ player.oa }} {{ player.full_name }},

dopo il crollo del Carducci pensavamo che gli ululati notturni fossero ormai un lontano ricordo, ma purtroppo ci sbagliavamo. Molti giovani coraggiosi sono andati a scavare fra le macerie del glorioso collegio ormai distrutto, in cerca di giochi da tavola che al Faedo latitano, ma non tutti sono tornati. Voci raccontano di corpi dilaniati ritrovati in via Kinzica, ma in effetti quelli erano comuni anche prima. Ho chiesto aiuto alle forze dell'ordine, ma tutti gli agenti sono impegnati a fissare i gradini delle chiese o le spallette del Lungarno, e nessuno ha raccolto il mio appello.

Come se ciò non bastasse, un'altra terribile minaccia aleggia su di noi. Persone silenziose, che restano nell'ombra di giorno ma che con il calare delle tenebre si manifestano, causando sconforto nei nostri cuori. La vita non sarà più la stessa, se non riusciremo a trovare un modo di fermare i santannini che ci impediscono di giocare a Subotto dopo la mezzanotte. La situazione è talmente fuori controllo che alcuni studenti sostengono di aver avvistato creature mostruose nei corridoi della Carovana, e giurano che no, non in tutti i casi si trattava di Zannier. A peggiorare ulteriormente le cose ci sono i ritrovamenti di candele accese e pentacoli rossi disegnati nel cortile del Faedo, ma ho parlato con gli admin di UZ e mi hanno assicurato che è tutto sotto controllo, servono per far funzionare le stampanti.

In ogni caso, tra lupi mannari e presunte attività ectoplasmatiche, la situazione è diventata insostenibile, e così ho preso una decisione sofferta ma che ritengo giusta: fare domanda di postdoc in Nuova Guinea, e tornare solo per la ventiquattr... ehm, le conferenze. Ma sono certo che gestirete la situazione brillantemente, come avete fatto gli scorsi anni con l'acquisto dei ricambi del Subot... no dunque, i nuovi giochi da tavola, tipo Pandemic che però già c'era... oppure dunque... insomma, sono sicuro che ve la caverete.

In attesa del mio trasferimento in Oceania, sto tenendo sotto controllo la situazione da Bruxelles. In mia assenza, vorrei affidare temporaneamente la carica di Sindaco ad una persona di cui mi fido ciecamente. Purtroppo nessuno di voi rispetta questo requisito, quindi ho deciso di lasciar perdere: in fondo il Sindaco non è mai servito a niente, potete farne a meno.

In bocca al Lupo,

\bigskip

\noindent Alessandro Iraci

\end{letter}

\end{document}



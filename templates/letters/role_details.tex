
Sei un contadino, ovvero un buzzurro, villano, bifolco, villico, bovaro, burino, eccetera. Hai essenzialmente due modalità di azione: ``influenzabile sempliciotto'', in cui credi supinamente a qualunque cosa ti dicano, e ``angry mob'', in cui brandisci un forcone e parti al linciaggio di chiunque ti capiti a tiro.

Il Contadino appartiene alla fazione dei Popolani e non ha alcun potere speciale.



Cerbiatti, coniglietti, allodole ed altri teneri animali del sottobosco per te non sono altro che selvaggina, cappotti, girocolli e scaldamani. E qualche volta un bel trofeo. Ti sei rifugiato in questo piccolo villaggio per sfuggire ad un gruppo di animalisti inferociti che aveva iniziato a pattugliare il tuo terreno di caccia preferito: la riserva naturale per la protezione del panda gigante.

Il Cacciatore appartiene alla fazione dei Popolani. Una sola volta durante l'arco della
partita, in una notte dopo la prima, egli può scegliere un personaggio vivo e puntarlo col fucile. Il Cacciatore uccide il personaggio scelto.



È giusto "custode" o "custode del cimitero"?



...

Il Divinatore appartiene alla fazione dei Popolani. All'inizio della partita è a conoscenza di quattro proposizioni, di cui esattamente due sono vere ed esattamente due sono false.

Le quattro proposizioni sono le seguenti.
\begin{quote}

    {{ message }} \\

\end{quote}






...


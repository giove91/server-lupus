\documentclass{letter}

\usepackage[utf8]{inputenc}
\usepackage[italian]{babel}


\signature{Giovanni Paolini}
\date{18 novembre 2014}
\begin{document}

\begin{letter}{}
\opening{Gentili concittadini,}

ma soprattutto gentilissim{{ player.oa }} {{ player.full_name }},

con sommo dispiacere sono costretto a comunicarvi che le strade e le dimore del nostro piccolo villaggio non sono più sicure.
Non si tratta più di pavidi santannini con catapulte di carta velina, ma di temibili e mostruose creature notturne.

Dopo gli accadimenti dello scorso marzo sembrava che la quiete fosse destinata a perdurare, ed ero perciò tornato dall'Australia a riprendere il mio mandato.
Tuttavia gli eventi delle ultime ore (specialmente la prematura morte di Michele Verde) mi spingono a un altro atto di eroico sacrificio: lascio tutto al capace {{ mayor.full_name }}, che da oggi è il nuovo Sindaco del villaggio.

È assolutamente necessario agire con efficacia e fermezza.
Giovanni Mascellani, Supremo Amministratore delle aule computer del villaggio, è dello stesso parere. Abbiamo pertanto deciso che fuggiremo entrambi in California, in modo da non interferire con le indagini.

Addio e buona Fortuna.

\closing{Il vostro precedente Sindaco,}

\end{letter}

\end{document}











mi duole comunicarvi che la quiete del nostro piccolo villaggio è stata compromessa da misteriosi attacchi di mostruose creature notturne.

Come se ciò non bastasse, l'ultima vittima è stata Giovanni Mascellani (il cui cadavere è stato ritrovato stamattina in sala server, orribilmente squartato e dilaniato) e ora non è rimasto nessuno che abbia voglia di amministrare l'aula computer.

È assolutamente necessario fronteggiare questa minaccia con efficacia e organizzazione, cosa che farete da soli perché ho deciso di fuggire in Australia.
Lascio tutto a {{ mayor.full_name }}, che da oggi è il nuovo Sindaco del villaggio.

Addio e buona Fortuna.

\closing{Il vostro precedente Sindaco,}

\end{letter}

\end{document}

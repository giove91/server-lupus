

\documentclass{letter}

\usepackage[utf8]{inputenc}
\usepackage[italian]{babel}
\usepackage{geometry}

\signature{Alessandro Iraci}
\date{1 marzo 2020}
\begin{document}

\begin{letter}{}
\opening{Gentili concittadini,}

ma soprattutto gentilissim{{ player.oa }} {{ player.full_name }},

come avremmo dovuto sospettare dal primo focolaio presente durante la 24 ore, l'epidemia che sta colpendo l'Italia è arrivata anche a Pisa. La triste novità è che purtroppo è arrivato anche il primo decesso nella nostra città. La vittima, i cui dati personali sono riservati, presentava chiari sintomi di infezione da COVID-19 quali tosse, difficoltà respiratorie, febbre, impronte di denti sul torace, e asportazione del muscolo cardiaco (probabilmente a morsi).

Per evitare il diffondersi della malattia, le autorità hanno immediatamente predisposto un blocco stradale e ferroviario, di fatto isolando la città. Con lo scopo di contenere gli attacchi di panico, è stato stabilito un servizio di vigilanza 24/7 di guardie armate di fucili a tranquillanti (per animali di grossa taglia): doveste vederne qualcuna in giro non preoccupatevi, è tutto normale. Infine, nel tentativo di limitare la diffusione del virus, sono state ordinate diverse tonnellate di legna da ardere e carbonella, utilizzabili per accendere fuochi su cui sterilizzare i sospetti contagiati.

Una recente ricerca scientifica sostiene che il COVID-19 possa causare, oltre ai sintomi descritti in precedenza, anche esperienze allucinatorie. Sembra infatti che diversi soggetti partecipanti ad uno studio a riguardo, inclusa la vittima di cui sopra, sostengano di aver ricevuto visite notturne da parte di lupi mannari o manifestazioni ectoplasmatiche di persone decedute: essendo i soggetti non in contatto fra di loro, la teoria più accreditata al momento è che il virus causi tali fenomeni psichici. Faccio pertanto appello al vostro senso civico: doveste anche voi riportare simili esperienze allucinatorie, vi esorto a presentarvi al centro sterilizzazioni in Piazza dei Cavalieri per impedire ulteriori contagi.

Vi invito ad usare massima cautela e razionalità: diversi soggetti, non volendo ammettere a sé stessi si essere stati contagiati, hanno cominciato a spargere voci su una presunta invasione di lupi mannari e negromanti che vogliono prendere il controllo della città. Ovviamente si tratta di \emph{fake news} e teorie complottiste prive di fondamento. La buona notizia è che la maggioranza dei soggetti in questione ha smesso di diffondere tali teorie dopo essere stata trattata con il processo di sterilizzazione precedentemente descritto.

Partirò domani stesso per Bruxelles per sottoporre la questione al Parlamento Europeo, nella speranza che lavorino ad una soluzione nell'interesse comune. Anche in questo caso, vi prego di non dare adito alle voci che sostengono che io mi rechi lì solo per ottenere documenti falsi con i quali fuggire in Canada in attesa che l'emergenza rientri: si tratta solo di dicerie senza alcuna base. Vi assicuro il massimo impegno per risolvere la situazione nel più breve tempo possibile. Nell'attesa, mi affido al vostro senso di responsabilità.

In bocca al Lupo,

\bigskip

\noindent Alessandro Iraci

\end{letter}

\end{document}


